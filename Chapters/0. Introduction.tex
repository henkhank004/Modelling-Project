
% - Explain the simplifying assumptions made in the simple SI-SI model.
% - Using the simple SI-SI model as a starting point, what other important factors do you think need
% to be considered in the model to make it more realistic? Which ones do you think are most
% important? How would these factors affect the outcome predicted by the model?
% - Pay particular attention and focus on those factors that you have chosen to address in Part II
% below. Explain why you choose to focus on these factors.
% II. Extended models (Individual Part) (1,5 points) [Weeks 2-4]
% - Maximum length in report: 1 page per student
% - From the factors you considered in Part I, each group member should choose one factor that
% they then consider in greater detail. The student should then develop and propose a modified
% or extended model (i.e., formulate it using a system of ODEs) that accounts for the factor you
% have chosen. Explain how you came up with the model as well as any assumptions made.
% Explain the meaning of all parameters and functions appearing in the model, and how these are


\section{Introduction} \label{introduction}
We will model a behaviour of a vector-borne disease. Firstly, we will introduce the SI-SI model and modify it using different extensions. Then we will choose one of them and (...)(to add later).
\subsection{SI-SI model}
We will now introduce the SI-SI model with its equations, assumptions and variables.
We have four compartments: $S(A)$ corresponding to the susceptible hosts, $I(A)$ with the infected hosts, $S(V)$ for the susceptible vectors and $I(V)$ for the infectious vectors. They move between them.
\subsubsection{Assumptions} \label{SISI-assumptions}
\begin{itemize}
    \item The total population is a constant $N$, $N = S(A)+I(A)$.
    \item There is no immunity.
    \item Humans and vectors are always born susceptible.
    \item Vectors do not recover from infection and remain infectious for life.
    \item On average, an infected person is infected for $d$ days.
\end{itemize}
\subsubsection{SI-SI model variables and parameters}
\begin{itemize}
    \item $S_h(t)$: Number of susceptible hosts at time $t$.
    \item $I_h(t)$: Number of infectious hosts at time $t$.
    \item $N_h(t) = S_h + I_h$: Total host population at time $t$.
    \item $N_v(t) = S_v + I_v$: Total vector population at time $t$.
    \item $S_v(t)$: Number of susceptible vectors at time $t$.
    \item $I_v(t)$: Number of infectious vectors at time $t$.
    \item $\beta_H$: Transmission rate from infected vectors to hosts.
    \item $\beta_V$: Transmission rate from infected hosts to vectors.
    \item $\gamma N_H$: Total number of bites per day. 
    \item $\gamma N_V$: Total number of times per day that a person is bitten. 
    \item $P_{V \to H}$: Probability that a single bite infects a person.
    \item $P_{H \to V}$: Probability that a bite infects a mosquito.
\end{itemize}
Let $\beta_H = \gamma N_V P_{V \to H} \; [\text{1/day}]$. Then the total number of infected hosts per unit time equals $\beta_H \left( \frac{I_V}{N_V} \right) S_H$, while the total number of infected vectors per unit time equals $\beta_V \left( \frac{I_H}{N_H} \right) S_V$.
\subsubsection{Final equations}
\begin{equation}
\frac{dS_H}{dt} = -\beta_{H} \frac{I_V}{N_V} S_H
\end{equation}
\begin{equation}
\frac{dI_H}{dt} = \beta_{H} \frac{I_V}{N_V} S_H
\end{equation}
\begin{equation}
\frac{dS_V}{dt} = -\beta_{V} \frac{I_H}{N_H} S_V
\end{equation}
\begin{equation}
\frac{dI_V}{dt} = \beta_{V} \frac{I_H}{N_H} S_V
\end{equation}
\subsection{Motivation for the chosen extensions}
We decided to add to the SI-SI model several extensions, each one individually. This enables a clear view on how the model is influenced. We start with studying the birth-and-death rate for humans and for vectors. We believe it is crucial to include them since births and deaths are an inseparable part of each population. We then consider the following: vaccination rate, immunity, host-to-host transmission, migration, behavioural changes of hosts and vectors, and, finally, seasonality. These extensions may not always appear, however, they are often observed in real-life situations. Hence, we find them highly interesting.

