\section{Group Model}
% Based on the individually proposed models above, choose one factor that you would like to
% consider as a group. As a group, develop and propose a modified model that accounts for the
% factor you have chosen. This can be a simplified version of one of the models, or one that
% combines two very simple models, or you can directly take one of the proposed models, and so
% on.
% - Explain the reason for your choice (e.g., relevance, feasibility, etc.).
% - Clearly explain how you came to the model. State any assumptions and simplifications, and
% explain the meaning of all parameters, functions, and variables in the system of ODEs.
% - How would you obtain reasonably realistic values or estimates for the constants needed in the
% model? Make sure to cite your sources. (Note: you will need this information for Part IV).
% - Nondimensionalize your model.

As a group, we decided to consider the \textbf{Immunity extension} which is introduced in section \ref{group}. We analyse the model as it is described in the individual part, without any modification.

\subsection*{Motivation}

When infected with a mosquito-borne disease, the immune system produces antibodies. This allows a person to gain immunity naturally. However, if there is no exposure to infection for a couple of years, the person can lose the immunity.

The chosen model includes this phenomenon as individuals immediately move from the \textit{infected} compartment to the \textit{(recovered) immune} one. Furthermore, after a certain time they return to the \textit{susceptible} compartment as they lose immunity.

Therefore, we believe that the extension is a relevant model for diseases such as malaria and dengue. In addition, thanks to its simplicity, it is feasible to analyse in this course. 

\subsection{Vector and Disease Characteristics Modeled}
We have chosen to focus on the model proposed by Dorottya Vaczy. This model describes the human (host) population dynamics using three compartments: Susceptible ($S_H$), Infected ($I_H$), and Recovered ($R_H$), with a flow $S_H \rightarrow I_H \rightarrow R_H \rightarrow S_H$ (hosts can move through susceptible, infected, and recovered states). For the mosquito (vector) population, the model employs a simpler Susceptible ($S_V$) to Infected ($I_V$) progression ($S_V \rightarrow I_V$), represented by the last two differential equations.
\\\\
Dengue fever presents complex immunological and transmission interactions that align pretty well with this mathematical model and its assumptions. Here we provide proof that dengue is well-suited to be modelled by this model:

\begin{itemize}
    \item \textbf{Vector-Borne Transmission}: Dengue is transmitted by female Aedes mosquitoes, primarily Aedes aegypti and, to a lesser extent, Aedes albopictus (https://www.who.int/emergencies/disease-outbreak-news/item/2023-DON498).
    \begin{itemize}
        \item \textit{Mathematical Representation}: This interaction is the cornerstone of the infection terms. Susceptible humans ($S_H$) become infected via contact with infected vectors ($I_V$), modeled by the term $\beta_H \frac{I_V}{N_V} S_H$ (affecting $\frac{dS_H}{dt}$ and $\frac{dI_H}{dt}$). Susceptible vectors ($S_V$) are infected by infected humans ($I_H$), modeled by $\beta_V \frac{I_H}{N_H} S_V$ (affecting $\frac{dS_V}{dt}$ and $\frac{dI_V}{dt}$).
        \item \textit{Alignment with Model Assumptions}: This biological reality directly validates \textbf{Model Assumption 3} which states: "$S_H$ gets infected by $I_V$ and $S_V$ by $I_H$." The model's core transmission mechanism correctly reflects Dengue's transmission pathway.
    \end{itemize}

    \item \textbf{Lifelong Infection in Mosquitoes}: Once a mosquito ingests blood from a viremic human and becomes infected with the Dengue virus, it generally remains infected and capable of transmitting the virus for the rest of its life (https://www.who.int/emergencies/disease-outbreak-news/item/2023-DON498). The virus replicates within the mosquito; it does not appear significantly detrimental, allowing persistent infection (https://www.sciencedirect.com/science/article/abs/pii/S2214574515000036).
    \begin{itemize}
        \item \textit{Mathematical Representation}: Susceptible vectors ($S_V$) move to the infected state ($I_V$) via the term $\beta_V \frac{I_H}{N_H} S_V$ and never go back nor die. 
        \item \textit{Alignment with Model Assumptions}: This persistence of infection in mosquitoes is precisely what \textbf{Model Assumption 2} ("as mosquitoes have a short lifespan, assume that they do not recover from the illness") captures. The biological fact of lifelong infection in Dengue vectors means the model's simplification of no recovery for $I_V$ is biologically accurate for this disease.
    \end{itemize}

    \item \textbf{Complex Host Immunity and Re-infection}: Dengue virus has four main serotypes (DENV-1 to DENV-4). Infection with one provides lifelong serotype-specific immunity (https://www.cdc.gov/dengue/about/index.html). However, cross-immunity to other serotypes after recovery is only temporary and partial (https://www.who.int/news-room/fact-sheets/detail/dengue-and-severe-dengue), lasting from months to a few years (e.g., 1–3 years) (https://pmc.ncbi.nlm.nih.gov/articles/PMC3730691/). After this, individuals are susceptible to other serotypes (https://www.who.int/news-room/fact-sheets/detail/dengue-and-severe-dengue).
    \begin{itemize}
        \item \textit{Mathematical Representation and Alignment with Model Assumptions}:
            \begin{itemize}
                \item \textit{Infection and Recovery Duration}: Humans become infected ($S_H \rightarrow I_H$) via $\beta_H \frac{I_V}{N_V} S_H$. They then recover ($I_H \rightarrow R_H$) at a rate $\frac{1}{d}$, meaning they spend an average of '$d$' days in $I_H$. This directly corresponds to \textbf{Model Assumption 4} ("an infected person is infected for $d$ days on average"). This '$d$' would represent the viremic period during which they are infectious and experience symptoms. The transition to $R_H$ also aligns with \textbf{Model Assumption 1} ("every host is able to recover... and then they gain immunity").
                \item \textit{Waning Immunity and Re-susceptibility}: The crucial Dengue feature of temporary cross-serotype immunity, followed by susceptibility to *other* serotypes, is captured by the model's $R_H \rightarrow S_H$ transition. This occurs at a rate $\frac{1}{r}$, meaning individuals spend an average of '$r$' days in the $R_H$ state before returning to $S_H$. This is precisely what is stated in \textbf{Model Assumption 5} ("a recovered person loses immunity after $r$ days on average and becomes susceptible again"). For Dengue, this assumption simplifies the complex multi-serotype reality into a single "recovered but temporarily immune" state ($R_H$) and a return to a general susceptible state ($S_H$), where '$r$' represents the average duration of this effective cross-protection.
            \end{itemize}
        \item \textbf{Generally Non-Fatal Nature of Mild Dengue}: While severe Dengue can lead to complications and mortality, classical Dengue fever is often self-limiting, with most individuals recovering within one to two weeks (\url{https://www.who.int/news-room/fact-sheets/detail/dengue-and-severe-dengue}).
    \begin{itemize}
        \item \textit{Mathematical Representation}: The model structure for hosts ($S_H, I_H, R_H$) inherently supports this. Infected individuals ($I_H$) transition to the recovered class ($R_H$) (term $+\frac{I_H}{d}$ in $\frac{dR_H}{dt}$). There is no explicit term for disease-induced death from the $I_H$ compartment, implying a constant total host population $N_H$.
        \item \textit{Alignment with Model Assumptions}: This characteristic aligns well with \textbf{Model Assumption 1} ("every host is able to recover from the illness...").
    \end{itemize}
    \end{itemize}
\end{itemize}

\subsection*{Realistic estimates}

\subsection*{Non-dimensionalization}

With the system of equations defined in [Dorottya's equations], we can define the independent and dependent quantities as described in [Dorrotya's part]:

\begin{equation}
(S_H \,\mathrm{[p]}, I_H \,\mathrm{[p]}, S_V \,\mathrm{[m]}, I_V \,\mathrm{[m]}, R_H \,\mathrm{[p]}) = f(t \,\mathrm{[d]}, N_H \,\mathrm{[p]}, N_V \,\mathrm{[m]}, \beta_H \, \mathrm{[d^{-1}]}, \beta_{V} \, \mathrm{[d^{-1}]}, r \, \mathrm{[d]}, d \, \mathrm{[d]}  )
\label{independent quantities}
\end{equation}

where the symbols in square brackets represent the units ($\mathrm{[p]}$,  $\mathrm{[m]}$, and $\mathrm{[d]}$ stand for the number of people, mosquitos, and days respectively). \footnote{Also note that in \cref{independent quantities} one of the independent variables is not truly independent since we can add the condition that $N_H + N_V = S_H + I_H + R_H + I_V + S_V = const.$, but that does not influence the analysis in any way here.} From \cref{independent quantities} we choose the set of dimensionally independent quantities as $\{N_H \,\mathrm{[p]}, N_V\,  \mathrm{[m]}, d \, \mathrm{[days]}\}$, and non-dimensionalise the variables in \cref{independent quantities} to obtain:
\begin{equation}
(s_h, i_h, s_v, i_v, r_v) = f(\tau, b_h, b_v, \rho)
\label{nondim quants}
\end{equation}

such that 

\begin{equation}
s_k = \frac{S_k}{N_k}, \quad i_k = \frac{I_k}{N_k}, \quad r_h = \frac{R_H}{N_H}, \quad \tau = \frac{t}{d}, \quad b_h = \beta_H d, \quad b_v = \beta_Vd, \quad \rho = \frac{r}{d}, \quad k \in \{H,V\}    
\end{equation}


substituting this into the original system from [dorottyas system] changes it to 

\begin{equation}
\left\{
\begin{aligned}
\displaystyle
\frac{ds_h}{d\tau} &= \dfrac{r_h}{\rho} \;-\; b_h\,s_h\,i_v,\\[1ex]
\frac{di_h}{d\tau} &= b_h\,s_h\,i_v \;-\; i_h,\\[1ex]
\frac{dr_h}{d\tau} &= i_h \;-\; \dfrac{r_h}{\rho},\\[1ex]
\frac{ds_v}{d\tau} &= -\,b_v\,i_h\,s_v,\\[1ex]
\frac{di_v}{d\tau} &= b_v\,i_h\,s_v.
\end{aligned}
\right.
\end{equation}

and using the fact that
\begin{equation}
\left\{
\begin{aligned}
        &s_h + i_h +r_h = 1 \\
        &s_v + i_v = 1
\end{aligned}
\right.
\end{equation}

since two more variables become dependent, we can simplify the system of equations to: 

\begin{equation}
\left\{
\begin{aligned}
\frac{ds_h}{d\tau}
  &= \dfrac{1 - s_h - i_h}{\rho}
   - b_h\,s_h\,i_v,\\[0.5ex]
\frac{di_h}{d\tau}
  &= b_h\,s_h\,i_v
   - i_h,\\[0.5ex]
\frac{di_v}{d\tau}
  &= b_v\,i_h\,(1 - i_v).
\end{aligned}
\right.
\end{equation}

% - Now that you have formulated a more realistic mathematical model, apply the theoretical
% concepts you learn in the course to investigate the behavior of the system.
% - When you apply the theoretical concepts and principles to your model, make sure to state it
% in your report! (E.g., “Applying Theorem X.Y to our model, ...”). This will be viewed very
% favorably when evaluating your report.
% - Here are some questions that might help you in your analysis:

% o (Tim) What are the possible equilibria of the model?
...
% o (Clem) Can you find first integrals or Lyapunov functions for your model?
...
% o (Maja) Can you find invariant subsets of the state space?
...
% o (Daan) Can you show global existence of solutions or blowup in finite time?
...
% o (Dorottya) Are there values of the system parameters for which the system has a stationary solution in the interior of the positive octant?
...
% o Can you say something about the stability of such a stationary solution?
...
% o What can you say about the stability of the disease-free equilibrium?
...
% o (Bartek) Are there solutions that correspond to an endemic equilibrium?
% (Bartek) Can you adjust the problem such that it does?
...
% Numerical implementation
% - (Vlad) Show by means of calculations how the system behaves. You should feel free to use built-in functions in mathematical software. (Note: If you think you will need some assistance in programming from the instructors or lecturers, you should use either Matlab or Mathematica. Otherwise, you should feel free to use other software packages such as Python.)
...
